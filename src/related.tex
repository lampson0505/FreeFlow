\section{Related Work} \label{sec:related}

%NetVM  \cite{netvm}
%DPDK and NetMap \cite{netmap}

\textbf{Inter-VM Communication:} 
The tussle between isolation and performance is not unique to containers.
In the field of Inter-VM Communication, there are several works that provides
high-performance by removing some of the isolation constraints.
For example, NetVM\cite{netvm} provides a shared-memory framework that
exploits the DPDK library to provide zero-copy delivery between VMs.
Netmap\cite{netmap} and VALE\cite{vale} (which is also used by ClickOS\cite{clickos}) 
are sharing buffers and metadata between kernel and userspace to eliminate memory copies.
However, these works actually strengthened the binding between VM and underlying host
structures, thus they cannot satisfy the portability requirement for \sysname. 
Also, the NetVM work is applicable only to intra-host setting,
constrained by the possibility of shared memory. Similarly, the Netmap
and VALE solutions are sub-optimal when the VMs/containers are located
on the same physical machine.

\textbf{RDMA and HPC}
RDMA originated from the HPC world, in the form of InfiniBand. The HPC community proposed RDMA enablement solutions for virtualization~\cite{ranadive2012toward} and containerization~\cite{rdmacontainers} technologies. Additionally, the HPC community have also been using shared-memory based communication~\cite{KNEM} for intra-node communication.

\textbf{General improvements}
%low latency socket

%http://facweb.cti.depaul.edu/jyu/Publications/Yu-Linux-TSM2004.pdf
%High performance networking using SR-IOV \cite{highsriov}