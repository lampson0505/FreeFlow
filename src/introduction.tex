\section{Introduction} \label{sec:introduction}

% What are containers and why they are becoming popular: pointing out portability and resource efficiency.

Containerization is a lightweight and efficient virtualization technology that wraps a piece of software in a complete filesystem cell which is called as a {\em container}. One container 
contains everything needed to run -- code, runtime, system tools, system libraries -- and guarantees that the software will always run the same on top of a proper host OS kernel. Containerization is rapidly
becoming popular for deploying applications in clouds, due to its good agility and portability of code deployment and high efficiency of resource sharing, compared with virtual machines (VMs)~\cite{?}.

% Current, container networking is only using IP networks. It has large overhead in resource efficiency and performance.
Containers need to communicate with each other via a {\em network channel} -- a software and hardware stack which can pass messages
from the senders to the receivers. 
Generally, the major concerns on a network channel can be summarized as four aspects:
\begin{itemize}
	\item{\textbf{Performance}}: the throughput and latency to send and receive data and messages.
	\item{\textbf{Isolation}}: the interferences between senders and receivers in resource, name space and security. 
	\item{\textbf{Potability}}: the independence between communications and the locations of senders and receivers.
	\item{\textbf{Overhead}}: the overhead on hardware resources to support the communications.
\end{itemize}
Currently, container runtime systems (e.g. Docker~\cite{?} and Kubernetes~\cite{?}) interconnect containers with TCP/IP network. 
They either leverage
network address translation (NAT) to map containers' ports to their hosts' ports (a.k.a. {\em host} mode), or use network bridges
and tunnels to provide a virtualized IP network that connects containers in one or multiple hosts (a.k.a. {\em bridge} and 
{\em overlay} modes). While TCP/IP network is an intuitive choice which offers isolation and portability for container communications, 
it heavily evolves the host OS' kernel stack and CPU which makes it suffer from huge performance and resource overhead.
For instance, the host OS kernel has to consume tremendous CPU cycles on jobs, such as address mapping, header manipulations and payload coping, etc., to process IP packets. It does not only waste the expensive computing resources in clouds, but also 
limits the network throughput and stretches the network latency between containers (see \S~\ref{sec:motivation}).

% Other options for networking.
Apart from TCP/IP networks, there are several other network channels that can potentially be used to enable 
container communications, while they all have their own pros and cons.
For instance, containers are essentially processes on the host OS, and they can communicate
via shared memory. The communications on this channel will bypass the entire host OS stack with negligible CPU overhead and achieve
excellent network performance in both throughput and latency. However, it requires containers to trust each other (weak isolation), 
and it has also weak
potability because only containers on the same physical host can potentially use this channel. Another option is remote direct access memory (RDMA) network stack. It does not only have good network performance and low CPU overhead but also good portability. 
However, it needs special hardware supports, and we also found that it is not efficient for containers in the same host to use RDMA
(see \S~\ref{sec:motivation}). 

Since different network channels achieves different trade-offs, how to choose the right network channel for containers depends 
on multiple factors, such as the applications' expectations, 
the hardware capability on the host, and the locations of containers, etc.. For example, if one application has a group of containers trusting each other, which means a sacrifice on isolation is tolerated, and it cares most about network performance, it had better use
shared memory to connect the containers on the same host, and RDMA on different hosts when the hardware is capable. However, the channel selection and switching processes should be transparent to the user code, because otherwise it will make the development of applications significantly more complicated.

% What we do
In this paper, we present~\sysname, a new abstraction layer for container networking to achieve the optimal network properties for different applications. 
As shown in Figure~\ref{??}, on one hand, \sysname has a network orchestrator in container runtime systems which decides, creates and manages the network channels according to factors as the containers' locations, software stack and hardware capability. One the other hand, \sysname provides a simple and uniform application programming interface (API) for applications. This API hides the heterogeneity of the under-layer channels used
to communicate with other containers, so it achieves the portability of containers.   

% What are the challenges
We are facing three main challenges in the design of~\sysname. First of all, how do we design the API layer. We have several options: (1) defining a brand new API set, which difficult not only because the complexity of designing a complete and widely accepted API, but also the training overhead to programmers; (2) re-using socket API of TCP/IP, which can hardly support the shared memory semantics; and
(3) finally we decide to adopt RDMA programming interfaces, Verbs, to achieve~\sysname's API layer. This is because Verbs inherently supports shared memory semantics, can be naturally extended to support multi-host case via RDMA, and has already been an industry standard and widely used by programmers. Additionally, there are also many libraries available to support TCP/IP on top of Verbs, which makes \sysname highly compatible with existing applications.

Next, \harry{how to select network channels.}

Finally, \harry{we should talk about the system challenges, but I do not know what exactly they are currently}.