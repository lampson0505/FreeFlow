\section{Discussions} \label{sec:discussion}

\textbf{Live migration:} 
\sysname could be a key enabler
for containers to achieve both high-performance and capability for live
migration. It will require the network library to interact with the orchestrator
more frequently, and may require maintaining additional per-connection state
within the library. We are currently investigating this further.

\textbf{Security and middle-box:} One valid concern for \sysname is how legacy
middle-boxes will work for communication via shared-memory or RDMA, and whether
security will be broken by using shared-memory or RDMA.  We do not yet have
complete answer to this issue. We envision that for security, \sysname would
only allow shared-memory among trusted containers, for example, container
belongs to the same vendor (e.g., running spark or storm).  We are investigating
how best to support existing middle-boxes (e.g. IDS/IPS) under \sysname.


\textbf{VM environment:}
So far our evaluation and prototype is based on containers running on
bare-metal.  But our design easily generalizes to containers deployed inside
VMs. Some issues, such as efficient inter-VM communication (perhaps using
NetVM~\cite{netvm}) need to be addressed, but we believe that it can be easily
done within the context of \sysname design.


\textbf{Scalability of \sysname :}

%\textbf{Network trouble shooting:}

%\textbf{VM environment:}
%So far our evaluation and prototype is based on containers running on bare-metal. 
%Although we believe our design is also general for containers deployed inside VMs,
%the VM environment can brought in further challenges such as how to handle the
%address mapping, how to enable shared-memory/RDMA or how to locate the containers
%and measure the utility of resources. We will address these challenges in our futureworks.
