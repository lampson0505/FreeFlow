\section{Background and Motivation}
\label{sec:motivation}

In this section, we present measurement results that demonstrates the network performance and resource overhead of different optional network channels. We focus on network throughput and latency when containers are located in the same or different hosts. 
We choose two kinds of hosts: physical machines and VMs in public clouds.

\subsection{Running environments of containers}

\begin{table} [t!]
\centering
\small
\begin{tabular}{ c || c | c | c | c }
  \hline
  Constraint & Case (a) & Case (b) & Case (c) & Case (d) \\ \hline \hline
  N/A & SharedMem & RDMA & SharedMem & RDMA \\ \hline
  w/o trust & TCP/IP & TCP/IP & TCP/IP & TCP/IP \\ \hline
  w/o RDMA NIC & SharedMem & TCP/IP & SharedMem & TCP/IP \\ \hline
\end{tabular}
\caption{\label{tab:best-network} The suggested network solution under different running environments in Figure~\ref{fig:deploy-cases} and constraints.}
\normalsize
\end{table}

Nowadays, a containerlized application is usually composed by multiple containers. For example, each master and slave node in Hadoop is an invividual container; A web service can include layers, such as load balancer, web server,
in-memory cache and backend database, and eacy layer can be a distributed 
system with multiple containerlized nodes. These containers are usually 
deployed into a multi-host server cluster, and the deployment is usually 
controlled by a cluster orchestrator, e.g. Mesos~\cite{?}. Working as a single 
application, the containers need to exchange data, and the network performance
has a huge impact on the overall application performance. 

Introduce the four representative cases of Figure~\ref{fig:deploy-cases}...

\subsection{The ineffiencies in container networking}


We measure the throughput and latency of TCP/IP, Shared-Mem and RDMA under cases
(a) (b) (c) and (d).

\harry{RDMA and Shared-Mem is missing under (c) and (d)}.


\para{The measurement setup}
(1) two bare metals: 40Gbps NICs.
(2) two VMs on top of the two bare metals.
(3) two VMs from Azure or EC2.

\subsection{Intra-host Network Performance}

\para{Throughput}

Three points: (1) TCP throughput is limited; (2) the bottleneck is CPU rather than memory bus for TCP/IP; (3) the bottleneck is NIC CPU for RDMA. (4) For intra-host cases, shared memory has the best performance.

Figure 1: throughput of single src-dst pair. Bar figure: x-axis: TCP/IP, RDMA and shared memory; y-axis: throughput;

Figure 2(a): throughput of multiple src-dst pair. Line figure: x-axis: number of pairs,; y-axis: throughput; Four lines: TCP/IP, RDMA, shared memory and memory bus.

Figure 2(b): CPU utilization. Line figure: x-axis: number of pairs,; y-axis: CPU utilization; Three lines: TCP/IP, RDMA, shared memory.

Figure 2(c): NIC CPU utilization. Line figure: x-axis: number of pairs,; y-axis: NIC CPU utilization; Three lines: TCP/IP, RDMA, shared memory.


\para{Latency}
Two points: (1) going through OS stack is increasing network latency; (2) The bottleneck is on system calls.

Figure 3: The stacked bar chart showing the total latency of TCP/IP, RDMA, shared memory and their components.