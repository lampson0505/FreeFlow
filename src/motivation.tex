\section{Background and Motivation}

\para{Containers}

Docker containers wrap a piece of software in a complete filesystem that contains everything needed to run: code, runtime, system tools, system libraries – anything that can be installed on a server. This guarantees that the software will always run the same, regardless of its environment.

Containers running on a single machine share the same operating system kernel; they start instantly and use less RAM. Images are constructed from layered filesystems and share common files, making disk usage and image downloads much more efficient.

Linux containers are getting popular. 

Nowadays, TCP/IP is the de facto protocol network communications, and most of applications
are built with APIs on top of TCP/IP. There are several advantages to use TCP/IP. First of all, 
portability ... . Second, isolation ... .

However, TCP/IP is sacrificing performance: bypassing OS kernels and memory copies.
The weakness in performance is getting more and more concerns, since morden applications
desire high bandwidth (e.g. 40Gbps) and low latency (e.g. < 10$\mu$s).
\harry{We show experiment results to show this point.}



RDMA is paid more attentions to solve this problem, however, RDMA is not good (or not the best) if the senders and receivers
are on the same physical machine. \harry{We show measurement results here to demonstrate how "bad"
RDMA is on the same host.}

The best solution for communication is shared memory if the senders and receivers are on the 
same physical machine: show measurement results.

Nonetheless, there are two concerns: security and portability.
For security, it is not a concern for most of the inter-container networking cases, since containers are created by the same user and basically trust each other.

For portability, we want to design a shim layer which offers a uniform programming interface to applications, as well as handle the heterogenousity of locations, under-layer software and hardware stacks.

